% Papel A4, fonte Times tamanho 12
\documentclass[a4paper,12pt]{article}

% Determinando tamanhos de margens
\usepackage[left=2.5cm, right=2cm, top=3cm, bottom=3cm]{geometry}

% Preâmbulo para documentos em português
\usepackage[brazilian]{babel}     % Idioma do texto (regras de hifenização e textos automáticos i.e. Figura, Tabela)
\usepackage[utf8]{inputenc}       % Codificação do texto (caracteres especiais)
\usepackage[T1]{fontenc}          % Operações de fontes (tipo, tamanho, etc.)
\usepackage{amsmath}
% Pacote para inclusão de imagens PNG, JPEG e PDF
\usepackage[pdftex]{graphicx}
%\usepackage{subfigure}            % Pacote necessário para geração de subfiguras
\usepackage{multirow}
% Cabecário com figura
\usepackage{fancyhdr}

\pagestyle{fancy}
\fancyhead{}     % Limpa cabeçalho
\fancyfoot{}     % Limpa rodapé
\fancyhead[L]{\includegraphics[width = 4cm]{../logo_simepar.png}}
\fancyhead[R]{\thepage}


\begin{document}

% Capa do relatório
\thispagestyle{empty}

\begin{flushleft}
\textbf{Sistema Meteorológico do Paraná --- SIMEPAR} \\
Curitiba --- PR, Caixa Postal 19100, CEP 81531-990, Tel/Fax: + 55 (41) 3320-2001
\end{flushleft}

\vspace{5.0cm}

\begin{center}
\fontsize{24pt}{28pt}\selectfont
\textbf{Relatório semestral de acompanhamento climatológico na região da UHE Baixo Iguaçu}\\
\vspace{18pt}
\textbf{Janeiro a junho de 2020
}\end{center}

\vspace{4.0cm}

\begin{center}
\textbf{José Eduardo Gonçalves}\\
Pesquisador\\

\vspace{12pt}

\textbf{Itamar Adilson Moreira}\\
Meteorologista\\
\end{center}

\vspace{12pt}

{\raggedleft \textbf{\textsc{Cesar Augustus Assis Beneti}}} \hfill \textsc{\textbf{Reinaldo Bonfim Silveira}}\\
\textit{Diretor Executivo} \hfill \textit{Coordenador de Modelagem Numérica}

\vspace{60pt}

\begin{center}
\textbf{Curitiba} \\
\textbf{Julho de 2021}
\end{center}

\newpage
\section{Introdução}

\hspace{0.5cm}Em atendimento ao Projeto Básico Ambiental da Usina Hidrelétrica Baixo Iguaçu, 
o presente relatório tem por finalidade fazer uma análise climatológica da região. Para tanto, as 
variáveis meteorológicas apresentadas nos relatórios mensais serão comparadas com o clima da região através de 
análise de anomalia das médias e dos desvios-padrão mensais.

Os parâmetros climatológicos --- média ($\overline{X}_C$) e desvio-padrão ($\sigma_{XC}$) 
esperados para cada variável meteorológica $X$ em 
cada mês na região --- foram calculados a partir de toda série diária histórica medida 
nas estações Cascavel (24535333), Nova Prata do Iguaçu (25345331), São Miguel do Iguaçu (25115408), 
Dois Vizinhos (25415305) e Planalto (23435346). 

Na sequência, para os meses em análise no presente relatório foram calculadas as médias mensais ($\overline{X}$) e os 
desvios-padrão ($\sigma_X$) (ambos baseados nos dados médios diários) para as variáveis meteorológicas medidas em cada posto. 
Desta forma, as anomalias das médias e dos desvios-padrão serão, respectivamente, dadas por
%
\begin{align*}
\overline{A}_{\mathrm{m\hat{e}s}} = \overline{X}_{\mathrm{m\hat{e}s}} - \overline{X}_{C_{\mathrm{m\hat{e}s}}} \\ 
\sigma_{A_\mathrm{m\hat{e}s}} = \sigma_{X_\mathrm{m\hat{e}s}} - \sigma_{XC_\mathrm{m\hat{e}s}} 
\end{align*}
%
em que $\overline{A}_\mathrm{m\hat{e}s}$ e $\sigma_{A_\mathrm{m\hat{e}s}}$  são as anomalias da média e do 
desvio-padrão para a variável meteorológica $X$ no mês em análise
no posto estudado. Esta análise será realizada para as seguintes variáveis: temperatura do ar média, máxima
e mínima ($^\circ$C), 
umidade relativa (\%), evapotranspiração (mm), radiação solar incidente (MJ m$^{-2}$), 
pressão atmosférica (hPa) e velocidade do vento (m/s).

Com relação à precipitação, a anomalia desta variável será analisada apenas com base no desvio da média do acumulado mensal. Neste sentido, 
com base nas séries históricas das cinco estações foi quantificada a precipitação acumulada mensal
esperada na região em cada mês. Desta forma, a anomalia será o desvio entre o acumulado no mês 
em análise e o acumulado esperado para este mês na região.

Por fim, a análise da direção do vento foi realizada apenas de forma qualitativa, uma vez que normalmente
ela é caracterizada por condições muito locais do relevo no entorno do posto de monitoranto. Sendo assim, será
apresentada uma tabela com a direção predominante em cada mês para cada estação, assim como a direção predominante no histórico 
de registros, para possibilitar a comparação entre o registrado e o esperado na região.



\newpage


        \section{Temperatura média do ar }
        \hspace{0.5cm} A seguir são exibidas as figuras das anomalias das médias e desvios-padrão da variável temperatura média do ar 
        registrada entre os meses de janeiro de 2020 e junho de 2020 na região da UHE Baixo Iguaçu.
        
        \begin{figure}[!htb]
        \centering
        \includegraphics[width=.7 \textwidth]{med_tmed.pdf}
        \caption{Anomalias na média da temperatura média do ar ($^\circ$C) no semestre analisado.}
        \label{fig:figmed_tmed}
        \end{figure}
        
        \begin{figure}[!htb]
        \centering
        \includegraphics[width=.7 \textwidth]{std_tmed.pdf}
        \caption{Anomalias no desvio padrão da temperatura média do ar ($^\circ$C) no semestre analisado.}
        \label{fig:figstd_tmed}
        \end{figure}  
        
        
        \newpage
        
        \section{Temperatura mínima do ar }
        \hspace{0.5cm} A seguir são exibidas as figuras das anomalias das médias e desvios-padrão da variável temperatura mínima do ar 
        registrada entre os meses de janeiro de 2020 e junho de 2020 na região da UHE Baixo Iguaçu.
        
        \begin{figure}[!htb]
        \centering
        \includegraphics[width=.7 \textwidth]{med_tmin.pdf}
        \caption{Anomalias na média da temperatura mínima do ar ($^\circ$C) no semestre analisado.}
        \label{fig:figmed_tmin}
        \end{figure}
        
        \begin{figure}[!htb]
        \centering
        \includegraphics[width=.7 \textwidth]{std_tmin.pdf}
        \caption{Anomalias no desvio padrão da temperatura mínima do ar ($^\circ$C) no semestre analisado.}
        \label{fig:figstd_tmin}
        \end{figure}  
        
        
        \newpage
        
        \section{Temperatura máxima do ar }
        \hspace{0.5cm} A seguir são exibidas as figuras das anomalias das médias e desvios-padrão da variável temperatura máxima do ar 
        registrada entre os meses de janeiro de 2020 e junho de 2020 na região da UHE Baixo Iguaçu.
        
        \begin{figure}[!htb]
        \centering
        \includegraphics[width=.7 \textwidth]{med_tmax.pdf}
        \caption{Anomalias na média da temperatura máxima do ar ($^\circ$C) no semestre analisado.}
        \label{fig:figmed_tmax}
        \end{figure}
        
        \begin{figure}[!htb]
        \centering
        \includegraphics[width=.7 \textwidth]{std_tmax.pdf}
        \caption{Anomalias no desvio padrão da temperatura máxima do ar ($^\circ$C) no semestre analisado.}
        \label{fig:figstd_tmax}
        \end{figure}  
        
        
        \newpage
        
        \section{Velocidade do vento }
        \hspace{0.5cm} A seguir são exibidas as figuras das anomalias das médias e desvios-padrão da variável velocidade do vento 
        registrada entre os meses de janeiro de 2020 e junho de 2020 na região da UHE Baixo Iguaçu.
        
        \begin{figure}[!htb]
        \centering
        \includegraphics[width=.7 \textwidth]{med_vento.pdf}
        \caption{Anomalias na média da velocidade do vento (m/s) no semestre analisado.}
        \label{fig:figmed_vento}
        \end{figure}
        
        \begin{figure}[!htb]
        \centering
        \includegraphics[width=.7 \textwidth]{std_vento.pdf}
        \caption{Anomalias no desvio padrão da velocidade do vento (m/s) no semestre analisado.}
        \label{fig:figstd_vento}
        \end{figure}  
        
        
        \newpage
        
        \section{Umidade relativa }
        \hspace{0.5cm} A seguir são exibidas as figuras das anomalias das médias e desvios-padrão da variável umidade relativa 
        registrada entre os meses de janeiro de 2020 e junho de 2020 na região da UHE Baixo Iguaçu.
        
        \begin{figure}[!htb]
        \centering
        \includegraphics[width=.7 \textwidth]{med_ur.pdf}
        \caption{Anomalias na média da umidade relativa (\%) no semestre analisado.}
        \label{fig:figmed_ur}
        \end{figure}
        
        \begin{figure}[!htb]
        \centering
        \includegraphics[width=.7 \textwidth]{std_ur.pdf}
        \caption{Anomalias no desvio padrão da umidade relativa (\%) no semestre analisado.}
        \label{fig:figstd_ur}
        \end{figure}  
        
        
        \newpage
        
        \section{Radiação solar incidente }
        \hspace{0.5cm} A seguir são exibidas as figuras das anomalias das médias e desvios-padrão da variável radiação solar incidente 
        registrada entre os meses de janeiro de 2020 e junho de 2020 na região da UHE Baixo Iguaçu.
        
        \begin{figure}[!htb]
        \centering
        \includegraphics[width=.7 \textwidth]{med_radiacao.pdf}
        \caption{Anomalias na média da radiação solar incidente (MJ/m$^2$) no semestre analisado.}
        \label{fig:figmed_radiacao}
        \end{figure}
        
        \begin{figure}[!htb]
        \centering
        \includegraphics[width=.7 \textwidth]{std_radiacao.pdf}
        \caption{Anomalias no desvio padrão da radiação solar incidente (MJ/m$^2$) no semestre analisado.}
        \label{fig:figstd_radiacao}
        \end{figure}  
        
        
        \newpage
        
        \section{Evapotranspiração }
        \hspace{0.5cm} A seguir são exibidas as figuras das anomalias das médias e desvios-padrão da variável evapotranspiração 
        registrada entre os meses de janeiro de 2020 e junho de 2020 na região da UHE Baixo Iguaçu.
        
        \begin{figure}[!htb]
        \centering
        \includegraphics[width=.7 \textwidth]{med_evapo.pdf}
        \caption{Anomalias na média da evapotranspiração (mm) no semestre analisado.}
        \label{fig:figmed_evapo}
        \end{figure}
        
        \begin{figure}[!htb]
        \centering
        \includegraphics[width=.7 \textwidth]{std_evapo.pdf}
        \caption{Anomalias no desvio padrão da evapotranspiração (mm) no semestre analisado.}
        \label{fig:figstd_evapo}
        \end{figure}  
        
        
        \newpage
        
        \section{Pressão atmosférica }
        \hspace{0.5cm} A seguir são exibidas as figuras das anomalias das médias e desvios-padrão da variável pressão atmosférica 
        registrada entre os meses de janeiro de 2020 e junho de 2020 na região da UHE Baixo Iguaçu.
        
        \begin{figure}[!htb]
        \centering
        \includegraphics[width=.7 \textwidth]{med_pressao.pdf}
        \caption{Anomalias na média da pressão atmosférica (hPa) no semestre analisado.}
        \label{fig:figmed_pressao}
        \end{figure}
        
        \begin{figure}[!htb]
        \centering
        \includegraphics[width=.7 \textwidth]{std_pressao.pdf}
        \caption{Anomalias no desvio padrão da pressão atmosférica (hPa) no semestre analisado.}
        \label{fig:figstd_pressao}
        \end{figure}  
        
        
        \newpage
        
        \section{Precipitação }
        \hspace{0.5cm} A seguir é exibida a anomalia dos acumulados mensais da variável precipitação registrados entre 
        os meses de janeiro de 2020 e junho de 2020 na região da UHE Baixo Iguaçu.
        
        \begin{figure}[!htb]
        \centering
        \includegraphics[width=.7 \textwidth]{med_chuva.pdf}
        \caption{Anomalias na média da precipitação (mm) no semestre analisado.}
        \label{fig:figmed_chuva}
        \end{figure}
        
        \newpage
        
\section{Direção do vento}

\hspace{0.5cm} Na sequência são exibidas as direções predominantes dos meses em análise, assim como as direções
climatológicas de cada posto, nas estações próximas à UHE Baixo Iguaçu. A estação UHE Baixo Iguaçu, por
ser recente e não possuir histórico de medições, apresenta apenas as direções predominantes registradas nos 
últimos meses.
 
\begin{table}[!hbt]
\begin{center}
\caption{Direções predominantes da velocidade do vento na região da UHE Baixo Iguaçu.}
\label{tab:dire}
\begin{tabular}{lccc}
\hline
Estação & Mês & Direção predominante & Direção predominante \\
        &     &      no mês          &      Climatológica   \\
\hline\multirow{6}{*}{Cascavel} & Janeiro & NE & NE \\
            & Fevereiro & NE & NE \\
            & Março & NE & NE \\
            & Abril & NE & NE \\
            & Maio & NE & NE \\
            & Junho & NE & NE \\
\hline\multirow{6}{*}{Dois Vizinhos} & Janeiro & S & S \\
            & Fevereiro & S & S \\
            & Março & S & S \\
            & Abril & S & S \\
            & Maio & S & S \\
            & Junho & S & S \\
\hline\multirow{6}{*}{Planalto} & Janeiro & SO & L \\
            & Fevereiro & SO & L \\
            & Março & SO & L \\
            & Abril & SO & L \\
            & Maio & SO & L \\
            & Junho & SO & L \\
\hline\multirow{6}{*}{São Miguel do Iguaçu} & Janeiro & SE & SE \\
            & Fevereiro & SE & SE \\
            & Março & SE & SE \\
            & Abril & SE & SE \\
            & Maio & SE & SE \\
            & Junho & N & SE \\
\hline\multirow{6}{*}{\textbf{UHE Baixo Iguaçu}} & \textbf{Janeiro} & \textbf{L} & \textbf{---} \\
            & \textbf{Fevereiro} & \textbf{L} & \textbf{{---}} \\
            & \textbf{Março} & \textbf{L} & \textbf{{---}} \\
            & \textbf{Abril} & \textbf{N} & \textbf{{---}} \\
            & \textbf{Maio} & \textbf{NE} & \textbf{{---}} \\
            & \textbf{Junho} & \textbf{L} & \textbf{{---}} \\
\hline
\end{tabular}
\end{center}
\end{table}


\end{document}